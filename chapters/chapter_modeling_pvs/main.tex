\chapter{Modeling Photovoltaics}\label{chapter:modeling_pvs}

In this chapter, \textbf{Modeling Photovoltaics}, we systematically review the
various types of abstractions in photovoltaics, starting from solar cells and
ending with solar arrays.

We start by observing how solar cell models can have differing granularities in
their composition and in how they address an array of internal qualities and
external environs that influence real world performance. Alongside defining
these models, we also propose modifications that may improve their accuracy and
precision, and consider tradeoffs that may occur in nonnominal conditions. After
defining these models, we then present an in-depth methodology for evaluating
them; we construct a dataset of solar cell \ac{I-V} curves characterized for use
on the \ac{LHRs} solar vehicle, then define techniques for extracting model
parameters for each cell. We proceed to use those model parameters to predict
their behavior in different conditions, and evaluate how they perform in terms
of model accuracy and precision.

We select the `best' set of solar cell models and use them as
building blocks to build larger models, namely those for solar modules. These
solar modules can take multiple shapes and sizes, and may exhibit reverse bias
behavior in the event of photovoltaic mismatch, a phenomenon caused by
nonuniform cells in series or in parallel. We also extend the module model by
adding a bypass diode in antiparallel, and discussing how solar cell reverse
bias behavior may drive their turn on conditions and mismatch mitigation
effects. From these solar module models, we formulate a metric to measure
mismatch, and propose suggestions and observations on how module size and cell
characteristics may influence the total efficiency of the module. Insights
developed in this section will later on become heuristics and algorithms for
optimizing module design in \autoref{chapter:optimizing_pvs}. We extend the test
methodology used for evaluating solar cells to solar modules, and validate
whether the module models pass muster.

% , either through manufacturing defects
% or lighting and thermal gradients across the module.

Finally, we take these solar module models and combine them together to form a
cohesive solar array model. From this model, we observe how the individual
module effects can generate a global \ac{I-V} curve with local and global
maxima, and discuss how this curve impacts the way the larger photovoltaic
system interacts with solar arrays. We also perform array testing using the
\ac{LHRs} solar array, and compare a simulated version of the array test with
real world data to determine a final, holistic evaluation of the models used.

\section{Modeling Solar Cells}\label{sec:modeling_solar_cells}

In this thesis, we will discuss three solar cell model abstractions:

\begin{itemize}
    \item Three Parameter Solar Cell Model (Single Diode Model)
    \item Five Parameter Solar Cell Model (Complete Single Diode Model)
    \item Seven Parameter Solar Cell Model (Double Diode Model)
\end{itemize}

To restrain the breadth of this document, we'll focus on the aforementioned
three models, since they are the most commonly cited and used abstractions. That
is not to diminish the dozens, if not hundreds more types of solar cell models,
like the three diode model proposed by Khanna et al~\cite{khanna_et_al}, or the
Bishop model with an avalanche breakdown
component~\cite{restrepo_cuestas_et_al}, or the \acf{DRM} that utilizes
antiparallel diode-resistor paths~\cite{restrepo_cuestas_et_al}.

\input{chapters/chapter_modeling_pvs/modeling_solar_cells/three_parameter_solar_cell.tex}
\newpage
\subsection{Five Parameter Solar Cell Model}\label{subsec:five_parameter_solar_cell_model}

\begin{figure}[h]
    \includegraphics[width=\textwidth]{solar_cell_five_parameter_model.png}
    \caption{Five Parameter, or Full Single Diode Model of a Solar Cell}
    \label{fig:single_diode_model_with_resistances}
\end{figure}

The most common model for solar cells is the five parameter solar cell model,
shown in \autoref{fig:single_diode_model_with_resistances}. This is the complete
form of the single diode model discussed in the previous section,
\autoref{subsec:three_parameter_solar_cell_model}. There are two added
components/parameters: a \acf{RS} and \acf{RSH}, whose primary roles are to
alter the shape of the knee-bend in the I-V curve. As such, this model improves
upon the main flaw of the three parameter solar cell model, that of poorly
predicting points clustering around the maximum power point.

In the following, we discuss the two added parameters and their specific effects
on the model.


\subsubsection{Shunt Resistance}\label{subsubsec:five_param_shunt_resistance}

\begin{figure}[h]
    \includegraphics[width=\textwidth]{series_shunt_resistance.png}
    \caption{Effect of Series (a) and Shunt Resistance (b) on \ac{I-V} Curve}
    \label{fig:series_shunt_resistance}
\end{figure}

As shown in \autoref{fig:series_shunt_resistance} (b) from Nelson~\cite{nelson},
as the \acf{RSH} decreases, the top of the knee-bend of the \acf{I-V} curve will
be forced down. At low values of \ac{RSH} (on the order of $10$ $\si{\ohm}$),
the knee-bend will be pushed down so much that the curve becomes a straight
line. At high values of \ac{RSH}, (on the order of $100$ $\si{\ohm}$), the curve
converges to some fixed maximum bend constrained by other parameters of the
model. This relationship is generally considered logarithmic.

The \acf{ISH} can be added to the simple form of the model as a new term as
shown in \autoref{eq:cell_output_current_3}. Assuming that the \acf{RS} is
negligible ($0$), we can determine that \ac{ISH} is a function of the \ac{RSH}
and the \acf{VL}, as shown in \autoref{eq:cell_output_current_4}.

\begin{equation}
    I_L = I_{PV} - I_D - I_{SH}
    \equnit{\si{\ampere}}
    \label{eq:cell_output_current_3}
\end{equation}

\begin{equation}
    I_L = I_{PV} - I_D - \frac{V_L}{R_{SH}}
    \equnit{\si{\ampere}}
    \label{eq:cell_output_current_4}
\end{equation}


\subsubsection{Series Resistance}\label{subsubsec:five_param_series_resistance}

The \acf{RS} forces the knee-bend of the \ac{I-V} curve to the left or right, as
opposed to up and down for \ac{RSH}. As \ac{RS} increases, more current is
consumed across the lumped resistance before reaching the terminals of the solar
cell, reducing the expected current in the curve as shown in
\autoref{fig:series_shunt_resistance} (a). At high values of \ac{RS}, the curve
likewise becomes a straight line.

The \ac{RS} term impacts the consumers of the five parameter solar cell model;
namely the \ac{ID} and the \ac{RSH} terms. A visualization of this is shown as
\autoref{fig:current_junction}.

\begin{figure}[!htbp]
    \centering
    \includegraphics[width=0.6\textwidth]{cell_kirchoff_current_junction.png}
    \caption{Current Flow Junction of Five Parameter Model Solar Cell}
    \label{fig:current_junction}
\end{figure}

Revisiting \autoref{eq:cell_dark_current_1}, we know that the dark current
depends on \acf{VL} generated by \acf{IL} flowing through equivalent \acf{RL} connected at the cell
terminals. This allows us to reformulate the dark current equation as
\autoref{eq:cell_dark_current_2}. Here, we add the voltage drop across the
lumped series resistance summed with the \ac{VL} to represent the
total voltage expected by the dark current model.

\begin{equation}
    I_D = I_0[\exp(\frac{V_L + I_L R_S}{V_T}) - 1]
    \equnit{\si{\ampere}}
    \label{eq:cell_dark_current_2}
\end{equation}

We can likewise use the voltage drop to update the \ac{ISH} term, shown
in \autoref{eq:cell_shunt_current}.

\begin{equation}
    I_{SH} = \frac{V_L + I_L R_S}{R_{SH}}
    \equnit{\si{\ampere}}
    \label{eq:cell_shunt_current}
\end{equation}

Combining these two effects, we form \autoref{eq:cell_output_current_5}.

\begin{equation}
    I_{L} =  I_{PV} - I_0[\exp(\frac{V_L + I_L R_S}{V_T}) - 1] - \frac{V_L + I_L R_S}{R_{SH}}
    \equnit{\si{\ampere}}
    \label{eq:cell_output_current_5}
\end{equation}

We note that this model is an implicit function and cannot easily (or prettily)
move all the \ac{IL} terms to the left side of the equation. As such, for these
types of problems, we will develop and use iterative solvers to determine
\ac{IL} for a given set of input parameters (\ac{RS}, \ac{G}, \ac{VL}, etc).
Iterative solvers involve starting with a guess for the output parameter (in
this case \ac{IL}) and attempt to improve upon that guess such that each side is
equal to each other or within some tolerance to each other. An in depth
discussion on how these solvers were implemented for this model and variants of
this model can be found in \autoref{appendix:iterative_solvers}.

\todo[inline]{Augment appendix note with reference to
\ref{subsubsec:modeling_solar_cell_datasets}. Relegate appendix note to discussion
about iterative solvers and steps to build iterative solver (Desmos -\> MATLAB -\>
Python).}


\subsubsection{Photocurrent as a Ratio of Shunt/Series Resistance}\label{subsubsec:photocurrent_shunt_series_relation}

An interesting addition to the five parameter cell model is presented by Cubas
et al~\cites{cubas_et_al,cubas_et_al_2}: they observe that
\autoref{eq:cell_output_current_5} in short circuit conditions results in
\autoref{eq:cell_short_circuit_current_6}.

\begin{equation}
    I_{SC} = I_{PV} - I_0[\exp(\frac{I_{SC} R_S}{V_T}) - 1] - \frac{I_{SC} R_S}{R_{SH}}
    \equnit{\si{\ampere}}
    \label{eq:cell_short_circuit_current_6}
\end{equation}

In their analysis of measurements taken across a broad spectrum of reference
solar cells, represented in \autoref{table:dark_current_reference}, the dark
current at short circuit conditions were well less than a single milliampere, an
insignificant fraction of the total operating current. From this observation
Cubas et al. rewrites the above expression to get the photocurrent as a function
of \ac{ISC} and a ratio of \ac{RS} and \ac{RSH}, shown in
\autoref{eq:cell_photocurrent_3}.

\begin{equation}
    I_{PV} = I_{SC}\frac{R_S + R_{SH}}{R_{SH}}
    \equnit{\si{\ampere}}
    \label{eq:cell_photocurrent_3}
\end{equation}

\begin{table}[h!]
    \begin{tabularx}{\textwidth}{
        | >{\raggedright\arraybackslash}X
        | >{\raggedright\arraybackslash}X
        | >{\raggedright\arraybackslash}X
        | >{\raggedright\arraybackslash}X
        | >{\raggedright\arraybackslash}X | }
        \hline
        Reference & Cell Type & \ac{ISC} (A) & $I_0[e^{\frac{I_{SC} R_S}{V_T}} -
        1]$ (A) & \ac{ID} / \ac{ISC} \\ \hline \hline
        Kennerud, 1969  & CdS   & 0.8040 & 1.56E-5 & 1.94E-5 \\ \hline
        Charles, 1981   & BSC   & 0.1023 & 2.21E-8 & 2.16E-7 \\ \hline
        Charles, 1981   & GSC   & 0.5610 & 1.01E-5 & 1.80E-5 \\ \hline
        Lo Brano, 2010  & Q6LM  & 7.6650 & 1.42E-9 & 1.85E-10 \\ \hline
    \end{tabularx}
    \caption{Dark Current Ratios for Various Reference Cells~\cite{cubas_et_al}}
    \label{table:dark_current_reference}
\end{table}

However, a cursory evaluation of the parameter space (\ac{VOC}, \ac{ISC},
\ac{G}, \ac{TC}, \ac{RS}, \ac{N}) reveals that the assumption that the dark
current is negligible breaks down when a subset of the following conditions
occur:

\begin{itemize}
    \item the \acf{VOC} becomes very small,
    \item the \acf{ISC} becomes very large,
    \item and the \acf{RS} becomes relatively large for some
    combination of \ac{VOC} and \ac{ISC}.
\end{itemize}

Is it to be noted that these parameters are tightly coupled, and therefore the
language specifying a parameter space upon which this term should be used
remains imprecise. We also note that \ac{TC} and \ac{N} when increased slightly
tighten the viable parameter space.

However, when considering a specific solar cell that is \textit{appropriate}
(e.g.\ it contains \ac{STC} defined parameters \ac{VOC} and \ac{ISC} with an
measured \ac{RS} that results in negligible \ac{ID}), this term remains
negligible unless the cell is exposed to (1) high temperatures or (2) high
intensity light, two conditions that tend to come hand in hand. These conditions
tend to only be experienced by concentrator photovoltaics and are highly
unlikely to be reached by normal solar cells.

We will observe later in \autoref{subsec:eval_solar_cell_models} that with our
dataset of Maxeon Gen III Bin Le1 solar cells, the vast majority of estimated
series resistance is well below $0.08 \si{\ohm}$, which results in dark currents
less than a m\si{\ampere}. This means that this modification (assuming it
improves the accuracy of the model), is well suited for our solar cells.

Incorporating this revision, we arrive at
\autoref{eq:cell_short_circuit_current_7}.

\begin{equation}
    I_L = I_{SC}\frac{R_S + R_{SH}}{R_{SH}} - I_0[\exp(\frac{V_L + I_L R_S}{V_T}) - 1] - \frac{V_L + I_L R_S}{R_{SH}}
    \equnit{\si{\ampere}}
    \label{eq:cell_short_circuit_current_7}
\end{equation}

\todo[inline]{See \url{https://www.desmos.com/calculator/nniw0mha2k} to play
around with the revised dark current model. Add as a figure later on compared to
experimental data.}


\subsubsection{Shunt and Series Resistance as a Function of Irradiance, Temperature}\label{subsubsec:rsh_rs_dependence}

Throughout this discussion, we introduced the notion of shunt and series
resistance as internal parasitics. However, we did not explore whether these
`internal parameters' are themselves affected by external conditions such as
irradiance and temperature.

A comprehensive review and experimental paper from Fébba et
al~\cite{febba_et_al} performed experiments on solar cells to evaluate the
effect of temperature and irradiance on shunt and series resistance, controlling
for the two independent variables in ranges of $25\si{\celsius}$ to
$55\si{\celsius}$ and $600\si{\watt/\meter^2}$ to $1000\si{\watt/\meter^2}$,
respectively. Four figures,
\autoref{fig:febba_shunt_resistance_and_temperature}~-~\autoref{fig:febba_series_resistance_and_irradiance}
are shown below to illustrate the following assertions.

For \acf{RSH}, they observed the following trends:

\begin{itemize}
    \item as temperature increases, the \ac{RSH} exponentially decays,
    \item and as irradiance increases, the \ac{RSH} linearly decreases.
\end{itemize}

For \ac{RS}, they observed the following trends:

\begin{itemize}
    \item as temperature increases, the \ac{RS} exponentially decays,
    \item and as irradiance increases, the \ac{RS} linearly increases.
\end{itemize}

\begin{figure}[!htbp]
    \centering
    \includegraphics[width=\textwidth]{febba_shunt_resistance_and_temperature.jpg}
    \caption{Shunt Resistance vs Temperature~\cite{febba_et_al}}
    \label{fig:febba_shunt_resistance_and_temperature}
\end{figure}

\begin{figure}[!htbp]
    \centering
    \includegraphics[width=\textwidth]{febba_shunt_resistance_and_irradiance.jpg}
    \caption{Shunt Resistance vs Irradiance~\cite{febba_et_al}}
    \label{fig:febba_shunt_resistance_and_irradiance}
\end{figure}

\begin{figure}[!htbp]
    \centering
    \includegraphics[width=\textwidth]{febba_series_resistance_and_temperature.jpg}
    \caption{Series Resistance vs Temperature~\cite{febba_et_al}}
    \label{fig:febba_series_resistance_and_temperature}
\end{figure}

\begin{figure}[!htbp]
    \centering
    \includegraphics[width=\textwidth]{febba_series_resistance_and_irradiance.jpg}
    \caption{Series Resistance vs Irradiance~\cite{febba_et_al}}
    \label{fig:febba_series_resistance_and_irradiance}
\end{figure}

Fébba et al. did not posit a revised model of the either resistance term
(although they did provide explanations on why the trends were reasonable), but
Baig et al.~\cite{baig_et_al} and MacAlpine et
Brandemuehl~\cite{macalpine_et_brandemuehl} introduced a variant of
\autoref{eq:series_resistance_1} that uses a \ac{ZETA}.

\begin{equation}
    R_S = R_{S,ref} \exp(\zeta [T_{C,ref} - T_C])
    \equnit{\si{\ohm}}
    \label{eq:series_resistance_1}
\end{equation}

We extend Fébba et al~\cite{febba_et_al}'s results to generate
\autoref{eq:series_resistance_2}, adding a \ac{ETA}, applied to
\autoref{eq:series_resistance_1}.

\begin{equation}
    R_S = R_{S,ref} \exp(\zeta [T_{C,ref} - T_C])[1 + \eta(G_{ref} - G)]
    \equnit{\si{\ohm}}
    \label{eq:series_resistance_2}
\end{equation}

We also propose \autoref{eq:shunt_resistance} to model the shunt
resistance, with \ac{KAPPA} and \ac{IOTA}.

\begin{equation}
    R_{SH} = R_{SH,ref} \exp(\kappa [T_{C,ref} - T_C])[1 + \iota [G_{ref} - G]]
    \equnit{\si{\ohm}}
    \label{eq:shunt_resistance}
\end{equation}

\autoref{eq:series_resistance_2} and \autoref{eq:shunt_resistance}'s
coefficients are not provided by the manufacturer, so they will have to be
estimated. We'll look at ways to measure series and shunt resistance in
\autoref{subsubsec:experimental_extraction_of_cell_parameters}, and how to take
advantage of our test setup to measure the coefficients. Additionally, we'll
attempt to replicate Fébba et al's work on a broader scale, with a temperature
and irradiance range of ($0\si{\celsius}$ to $100\si{\celsius}$
and $0\si{\watt/\meter^2}$ to $1000\si{\watt/\meter^2}$, respectively).


\subsubsection{Non Uniform Series Resistance}\label{subsubsec:nonuniform_series_resistance}

As an aside to this thesis, we note that the solar cell models described assume
that the cell is uniform in composition and thus can represent the series
resistance as a lumped resistance. In actuality, the cell is a two (actually
three, but for all intents and purposes the thickness is irrelevant in terms of
affecting the series resistance -although it may affect the shunt resistance)
dimensional network of resistors and diodes
(\autoref{fig:cell_with_varying_series_resistance}).

\begin{figure}[!htbp]
    \centering
    \includegraphics[width=0.6\textwidth]{cell_with_varying_series_resistance.png}
    \caption{Solar Cell With Varying Series Resistances~\cite{pveducation_measurement_of_series_resistance}}
    \label{fig:cell_with_varying_series_resistance}
\end{figure}

If the cell series resistance was measured using two probes at various points in
the cell, we would likely see that the places of smallest resistance will focus
on the direct paths between two terminals; the places of largest resistance will
be at the edges of the cell where the current path is longest. This is
visualized on a Maxeon Gen III cell in
\autoref{fig:maxeon_gen_iii_cell_current_resistance_path}, where the darker
paths represent higher resistances.

\begin{figure}[!h]
    \centering
    \includegraphics[width=0.88\textwidth]{maxeon_gen_iii_cell_current_resistance_paths.png}
    \caption{Current Paths of Maxeon Gen III Cell}
    \label{fig:maxeon_gen_iii_cell_current_resistance_path}
\end{figure}

This series resistance non uniformity becomes more important to the
resultant \ac{I-V} curve in low light conditions. Under uniform, bright
conditions, current from the photovoltaic effect is generated evenly and
conducts to the contacts regardless of the resistance along the paths. In the
dark, any current that has to flow through the cell (either from a partially
unshaded region or from an external source) will favor the shortest/least
resistive paths. As such, we would expect the apparent series resistance to vary
in various lighting contexts.

For the purposes of this thesis, we will continue to assume that the solar cell,
the base unit of our model, is uniform in internal and external characteristics.
However, we will look at intercell variance in our module models. Further
research into intracell series variance is explored by Bowden et
Rohatgi~\cite{bowden_et_rohatgi}.


\subsubsection{Model Summary}\label{subsubsec:five_param_model_summary}

To conclude this discussion, we will review the components that make up the
five parameter cell model, propose an item of further exploration, and propose a
complete model function that incorporates the topics discussed.

Firstly, the five parameter cell model retains the attributes of the three
parameter cell model, being the complete form of the single diode model. It adds
two parameters, a \acf{RSH} and \acf{RS} that represent ohmic losses in the
solar cell, which primarily affect the knee-bend of the resultant \ac{I-V}
curve. These two parameters help reduce error in the model around the knee-bend
that cannot fully be represented by the ideality factor. However, these
additions increase the complexity of the model, and the resultant form is an
implicit equation that requires an iterative solver approach.

Secondly, we investigate a revision to the photocurrent model to make it also a
function of \ac{RS} and \ac{RSH}. This was obtained by evaluating the short
circuit condition of the existing model and reducing the dark current term under
appropriate conditions. We note that this new model may not work under specific
conditions, namely for concentrator solar cells or for solar cells with
inordinately large series resistance relative to their specific \ac{VOC} and
\ac{ISC} combination.

We also discuss evaluating \ac{RS} and \ac{RSH} themselves as a function of
temperature and irradiance. We observe that these values tend to have
exponential relationships with temperature and linear relationships with
irradiance, although we require further data to validate the strength of these
correlations. We derive initial models for these parameters, and discuss real
world conditions in which they might deviate from our expectations (e.g. partial
shading). As such, we will revisit both of these modifications to the base model
in a further discussion to prove or disprove their veracity and usefulness to
the overall model.

Finally, we incorporate these changes into the complete function defined in the
previous sections. This is presented as \autoref{eq:cell_output_current_6}
(\ac{ISC}, \ac{VOC}, \ac{RS}, \ac{RSH}, and \ac{VT} abstracted out for clarity
and brevity). We observe that this complete model builds upon the existing
parameters named in \autoref{subsubsec:three_param_model_summary} by adding two
extra reference parameters:

\begin{itemize}
    \item \acf{RSREF}
    \item \acf{RSHREF}
\end{itemize}

and four more curve fitting parameters:

\begin{itemize}
    \item \acf{ZETA}
    \item \acf{ETA}
    \item \acf{KAPPA}
    \item \acf{IOTA}
\end{itemize}

Likewise with the \acf{N} and \acf{GAMMA} discussed in the three parameter solar
cell model, we will look at estimating the four new curve fitting parameters
using curve fitting and other statistical techniques. We can potentially
establish known thermal coefficients (with some error) for these cells using the
\ac{CLT}, and customize each cell with only the following variables: \ac{RSREF},
\ac{RSHREF}, which can be determined empirically using a single measurement at
\ac{STC}.


\begin{equation}
    \begin{split}
        I_L(V_L, G, T_C) &= I_{PV}(G, T_C, R_S, R_{SH}) - I_D(V_L, G, T_C, R_S) - I_{SH}(R_S, R_{SH}) \\
        & = I_{SC}(G, T_C)\frac{R_S + R_{SH}}{R_{SH}} - I_0(G, T_C)[\exp(\frac{V_L + I_L R_S}{V_T(T_C)}) - 1] - \frac{V_L + I_L R_S}{R_{SH}} \\
        & = I_{SC}(G, T_C)\frac{R_S + R_{SH}}{R_{SH}} - I_{SC}(G, T_C)\frac{\exp(\frac{V_L + I_L R_S}{V_T(T_C)}) - 1}{\exp(\frac{V_{OC}(G, T_C)}{V_T(T_C)}) - 1} - \frac{V_L + I_L R_S}{R_{SH}} \\
        & = I_{SC}(G, T_C)[\frac{R_S + R_{SH}}{R_{SH}} + \frac{1 - \exp(\frac{V_L + I_L R_S}{V_T(T_C)})}{1 - \exp(\frac{V_{OC}(G, T_C)}{V_T(T_C)})}] - \frac{V_L + I_L R_S}{R_{SH}}
    \end{split}
    \equnit{\si{\ampere}}
    \label{eq:cell_output_current_6}
\end{equation}

\todo[inline]{See \url{https://www.desmos.com/calculator/yp0rhmabkz} to play
around with the complete five parameter solar cell model. Add as a figure later
on compared to experimental data.}

\newpage
\subsection{Seven Parameter Solar Cell Model}\label{subsec:seven_parameter_solar_cell_model}

\begin{figure}[h]
    \includegraphics[width=\textwidth]{solar_cell_seven_parameter_model.png}
    \caption{Seven Parameter, or Double Diode Model of a Solar Cell}
    \label{fig:double_diode_model}
\end{figure}

The seven parameter solar cell model (\autoref{fig:double_diode_model}), also
known as a double diode model, builds upon the five parameter model by
introducing a second diode (hence the name) to more accurately model internal
current losses.

These losses can be split into the following:

\begin{itemize}
    \item losses due to carrier recombination in the space charge region of the
    P-N junction,
    \item and losses due to surface recombination.
\end{itemize}

These currents are denoted as \acf{ID1} and \acf{ID2}, respectively. By
differentiating between the two primary recombination processes in the cell, the
seven parameter model is generally considered more accurate than the five
parameter model.

The general form of this equation is shown in \autoref{eq:cell_output_current_7}.

\begin{equation}
    I_L = I_{PV} - I_{D1} - I_{D2} - I_{SH}
    \equnit{\si{\ampere}}
    \label{eq:cell_output_current_7}
\end{equation}

This results in the \autoref{eq:cell_output_current_8} when all components have
been inserted:

\begin{equation}
    \begin{split}
        I_L(V_L, G, T_C) &= I_{PV}(G, T_C, R_S, R_{SH})
                          - I_{D1}(V_L, G, T_C, R_S)
                          - I_{D2}(V_L, G, T_C, R_S) \\
        & \quad           - I_{SH}(R_S, R_{SH}) \\
        &                 = I_{SC}(G, T_C)\frac{R_S + R_{SH}}{R_{SH}}
                          - I_{01}(G, T_C)[\exp(\frac{q[V_L + I_L R_S]}{n_1 k_B T_C}) - 1] \\
        & \quad           - I_{02}(G, T_C)[\exp(\frac{q[V_L + I_L R_S]}{n_2 k_B T_C}) - 1]
                          - \frac{V_L + I_L R_S}{R_{SH}} \\
        &                 = I_{SC}(G, T_C)\frac{R_S + R_{SH}}{R_{SH}}
                          - I_{SC}(G, T_C)\frac{\exp(\frac{q[V_L + I_L R_S]}{n_1 k_B T_C}) - 1}{\exp(\frac{qV_{OC}(G, T_C)}{n_1 k_B T_C}) - 1} \\
        & \quad           - I_{SC}(G, T_C)\frac{\exp(\frac{q[V_L + I_L R_S]}{n_2 k_B T_C}) - 1}{\exp(\frac{qV_{OC}(G, T_C)}{n_2 k_B T_C}) - 1}
                          - \frac{V_L + I_L R_S}{R_{SH}} \\
        &                 = I_{SC}(G, T_C)[
                                \frac{R_S + R_{SH}}{R_{SH}}
                              + \frac{1 - \exp(\frac{q[V_L + I_L R_S]}{n_1 k_B T_C})}{1 - \exp(\frac{qV_{OC}(G, T_C)}{n_1 k_B T_C})} \\
        & \quad               + \frac{1 - \exp(\frac{q[V_L + I_L R_S]}{n_2 k_B T_C})}{1 - \exp(\frac{qV_{OC}(G, T_C)}{n_2 k_B T_C})}]
                          - \frac{V_L + I_L R_S}{R_{SH}}
    \end{split}
    \equnit{\si{\ampere}}
    \label{eq:cell_output_current_8}
\end{equation}

We note in this equation \ac{VT} was substituted back in to demonstrate that
each ideality constant for each diode is unique.

\newpage
\todo[inline]{Behold! True evil!!!\newline Not for general consumption.}
\begin{equation}
    \begin{split}
        I_L(V_L, G, T_C) &= I_{SC,ref}\frac{G}{G_{ref}}[1 - \alpha(T_{C,ref} - T_C)] \\
        & \quad             [ \\
        & \quad\quad            \frac{R_{S,ref} \exp(\zeta [T_{C,ref} - T_C])[1 + \eta(G - G_{ref})]}{R_{SH,ref} \exp(\kappa [T_{C,ref} - T_C])[1 - \iota(G - G_{ref})]} + 1 \\
        & \quad\quad          + \frac{1 - \exp(\frac{q[V_L + I_L R_{S,ref} \exp(\zeta [T_{C,ref} - T_C])[1 + \eta(G - G_{ref})]]}{n_1 k_B T_C})}{1 - \exp(\frac{q[V_{OC,ref}[1 - \beta (T_{C,ref} - T_C)] + \frac{nk_B(T_{C,ref} + T_C/\gamma)}{q}\ln(\frac{G}{G_{ref}})](G, T_C)}{n_1 k_B T_C})} \\
        & \quad\quad          + \frac{1 - \exp(\frac{q[V_L + I_L R_{S,ref} \exp(\zeta [T_{C,ref} - T_C])[1 + \eta(G - G_{ref})]]}{n_2 k_B T_C})}{1 - \exp(\frac{q[V_{OC,ref}[1 - \beta (T_{C,ref} - T_C)] + \frac{nk_B(T_{C,ref} + T_C/\gamma)}{q}\ln(\frac{G}{G_{ref}})](G, T_C)}{n_2 k_B T_C})} \\
        & \quad             ] \\
        & \quad             - \frac{V_L + I_L R_{S,ref} \exp(\zeta [T_{C,ref} - T_C])[1 + \eta(G - G_{ref})]}{R_{SH,ref} \exp(\kappa [T_{C,ref} - T_C])[1 - \iota(G - G_{ref})]}
    \end{split}
    \equnit{\si{\ampere}}
    \label{eq:cell_output_current_9}
\end{equation}

\todo[inline]{See \url{https://www.desmos.com/calculator/69rs9uo14f} to play
around with the complete seven parameter solar cell model. Add as a figure later
on compared to experimental data.}


\subsubsection{Model Summary}\label{subsubsec:seven_param_model_summary}

\todo[inline]{Might want to look for some more novel content, or wrap this section up as
is. Nothing particularly new here besides another parameter to estimate.}



\newpage
\subsection{Evaluation of Solar Cell Models}\label{subsec:eval_solar_cell_models}

To evaluate these solar cell models and their proposed modifications, we used a
set of almost 450 Maxeon Gen III and Maxeon C60 solar cells. In the following,
we discuss the aspects of this collection, how we characterized the solar
cells to generate a robust dataset with custom \acf{HW} and \acf{SW}, and
techniques (both experimental and statistical) to evaluate and estimate their
parameters. Finally, we'll use these parameters to compare the models and their
real world equivalents to determine model accuracy and precision.

\subsubsection{Solar Cell Dataset}\label{subsubsec:solar_cell_dataset}

The solar cells used for the \ac{LHRs} solar vehicle are a mixture of Maxeon Gen
III and Maxeon C60 solar cells. These solar cells were selected primarily due to
financial and availability constraints; historically, in the last two solar
vehicle revisions (2018, 2021) Gen III Bin Le1 cells have been used, but this
year the team decided to procure cheaper, more easily available C60 cells from
secondary suppliers. Regardless, these cell lines remain state of the art
despite their age\footnote{the dates are unclear, but it appears that C60 was
introduced around 2007~\cite{sunpower_history} and the Gen III has been around
as long as 2013~\cite{smith_et_al}.}; both Aptera
Motors~\cite{aptera_solar_cells} and Lightyear
One~\cite{lightyear_one_solar_cells} -the latter of which is a former
competitive solar vehicle team- have announced cooperation with Maxeon to use
their solar cells. Aptera in particular uses the Gen III
cells~\cite{aptera_solar_cells}.

While these cell types are both $125 \si{\mm}$ by $125 \si{mm}$ (see
\autoref{fig:maxeon_gen_iii_cell_footprint} for a visualization of the cell
physical layout), the Gen III cells are slightly more efficient than the C60
cells. Their (Gen III) rear contacts also tend to be slightly narrower than the
C60 cells. Their electrical characteristics are outlined in
\autoref{fig:maxeon_gen_iii_cell_characteristics} and
\autoref{fig:maxeon_c60_cell_characteristics}. Note that the Maxeon Gen III
cells are explicitly Bin Le1 cells, although the dataset will later show that
the binning for both groups of cells tends to not be very respective of the
actual measured \ac{I-V} curves, which is likely due to the variance in our
testing setup.

Since these cells were unpacked and designated for specific years,
\autoref{table:solar_cell_dataset} is provided to delineate between the
different types and `lines' of cells tested. `Lines' in this sense indicate
the academic year the cells were originally unpacked and tested.

\begin{figure}[!htbp]
    \includegraphics[width=\textwidth]{maxeon_gen_iii_cell_footprint.png}
    \caption{Maxeon Gen III Cell Footprint}
    \label{fig:maxeon_gen_iii_cell_footprint}
\end{figure}

\begin{figure}[!htbp]
    \centering
    \includegraphics[width=0.95\textwidth]{maxeon_gen_iii_cell_characteristics.png}
    \caption{Maxeon Gen III Cell Characteristics}
    \label{fig:maxeon_gen_iii_cell_characteristics}
\end{figure}

\begin{figure}[!htbp]
    \centering
    \includegraphics[width=0.95\textwidth]{maxeon_c60_cell_characteristics.png}
    \caption{Maxeon C60 Cell Characteristics}
    \label{fig:maxeon_c60_cell_characteristics}
\end{figure}

\begin{table}[!htbp]
    \begin{tabularx}{\textwidth}{
        | >{\raggedright\arraybackslash}X
        | >{\raggedright\arraybackslash}X
        | >{\raggedright\arraybackslash}X
        | >{\raggedright\arraybackslash}X | }
        \hline
        Cell Line   & Year Unpacked & Type      & Number of Cells \\ \hline \hline
        RP          & 2022          & C60       & X               \\ \hline
        MW          & 2020          & Gen III   & X               \\ \hline
        2019\_Le1   & 2019          & Gen III   & X               \\ \hline
        BU          & 2018          & Gen III   & X               \\ \hline
    \end{tabularx}
    \caption{Cell Lines Used in Solar Cell Dataset}
    \label{table:solar_cell_dataset}
\end{table}
\todo[inline]{Add number of cells tested to each group in table.}


\subsubsection{Solar Cell Test Setup}\label{subsubsec:solar_cell_test_setup}

\begin{figure}[!htbp]
    \includegraphics[width=\textwidth]{cell_test_setup.png}
    \caption{Photovoltaic Testing Setup}
    \label{fig:cell_test_setup}
\end{figure}

To characterize solar cells, we developed a test setup as outlined in
\autoref{fig:cell_test_setup}. In this test setup, we maintain three critical
requirements:

\begin{itemize}
    \item The test article experiences irradiance and temperature that is
    \textit{temporally} and \textit{spatially} uniform.
    \item The test article experiences a \textit{measurable} irradiance and
    temperature.
    \item The irradiance and temperature experienced by the test article can be
    physically manipulated.
\end{itemize}

To achieve these aforementioned requirements, we first use a solar simulator
consisting of a set of grow light modules (MPJA 34769-OPs) mounted to an
aluminum plate heatsink. The MPJA grow light modules have an emittance
spectra as shown in \autoref{fig:grow_light_spectra}; compared to the AM1.5
solar spectra (in particular, ASTMG173) in \autoref{fig:solar_spectra}, it can
be said that these LEDs are not a great characterization of natural sunlight.
A proposed design of a multi-channel LED based solar simulator is presented in
\autoref{appendix:solar_simulator}, following after similar efforts by
others~\cites{lopez_fraguas_et_al,plyta_et_al,al_ahmad_et_al,naskari_et_al},
but that is beyond the scope of this thesis.

\begin{figure}[!htbp]
    \centering
    \includegraphics[width=0.65\textwidth]{solar_spectra.png}
    \caption{AM0, AM1.5 Solar Spectra}
    \label{fig:solar_spectra}
\end{figure}

\begin{figure}[!htbp]
    \centering
    \includegraphics[width=0.65\textwidth]{mpja_grow_light_spectra.png}
    \caption{MPJA Grow Light Spectrum}
    \label{fig:grow_light_spectra}
\end{figure}

\begin{figure}[!htbp]
    \centering
    \includegraphics[width=0.65\textwidth]{tsl2591_spectral_responsivity.png}
    \caption{TSL2591 Spectral Responsivity}
    \label{fig:tsl2591_spectral_responsivity}
\end{figure}
\todo[inline]{Add reference to AMS TSL2591 datasheet. Figure 11.}

To ensure that the solar simulator irradiance is temporally uniform, a low
cost luminosity sensor (TSL2591) is used to measure the irradiance over a fixed
period of time. This period of time should be long enough to determine whether
the lights have a warm up time and change in irradiance over the expected
experiment duration. In order to get irradiance, we must convert the TSL2591
`counts' into $\si{watt}/\si{\meter}^2$; this is not straightforward, since
the normalized responsivity spectrum of the TSL2591
(\autoref{fig:tsl2591_spectral_responsivity}) is also quite divergent from
AM1.5G solar spectrum. It is, however, relatively close to the absorbance
spectra observed by the Maxeon Gen III solar cells, so the irradiance measured
by the device will slightly undershoot the expected value captured by thebcell.
A further discussion on methods to calibrate the sensor readings for a true
observed solar spectra is presented in \autoref{appendix:tsl2591_calibration}.
These calibration methods are useful for testing photovoltaics outdoors.

To ensure the solar simulator irradiance is spatially uniform, the lighting
modules relative to each other and relative to the plate need to be spaced
appropriately. The light modules have a nonuniform intensity profile (e.g. light
is concentrated radially from the center of the fixture), and thus require some
overlap in illuminance area to create a superimposed, roughly uniform light
distribution. The spacing is empirically evaluated by also using the TSL2591,
similar to how \acf{PPFD} is measured~\cite{ppfd_measurement}: a closely spaced
set of points is mapped to their respective intensity measurements to
determine the variance in intensity and the lights are moved closer/farther
apart accordingly to minimize said variance.

The photovoltaic, a solar cell or solar module of up to $500 \si{\mm}$ by $250
\si{mm}$ (equivalent to 4 cells by 2 cells) in size, is placed upon a thermal
bed separated by a thin, electrically insulating layer of Kapton tape; this
thermal bed maintains the photovoltaic surface temperature via conductance, and
pumped distilled water is circulated through the bed by a \todo{Insert name of
heater/ chiller device} XXX.

After controlling for the light intensity and temperature of the test article,
the actual measurement of the solar cell parameters is performed by custom
\acp{PCB} developed by the team. The primary \ac{PCB} measures the \ac{I-V}
curve of the photovoltaic by adjusting the perceived load across the terminals.
It does this by actuating a pair of high power \acfp{MOSFET}, particularly in
the ohmic region between open and short circuit. Small steps to the gate voltage
combined with a current and voltage sensor allow for high resolution
measurements of the test article. These measurements are communicated back to
the user via \ac{USB} and captured using the Python scripting language. This
allows us to communicate to the device to set measurement profiles using either
a \ac{CLI} or \ac{GUI}. A secondary \ac{PCB} containing the TSL2591 is also
hooked up to the primary \ac{PCB} via a \ac{CAN} hardware interface. This
allows us to also combine irradiance measurements with the electrical
measurements and appropriately obtain predicted parameters at \ac{STC}. A
further description of the hardware and software implementation of these two
\acp{PCB} are provided in \autoref{appendix:curve_tracer_design} and
\autoref{appendix:blackbody_design}.

These elements of the test setup are depicted in \autoref{fig:test_setup}.

\begin{figure}[!htbp]
    \centering
    \includegraphics[trim={2cm 10cm 0cm 3cm},clip,width=\textwidth]{test_setup.jpg}
    \caption{Photovoltaic Testing Setup}
    \label{fig:test_setup}
\end{figure}


\subsubsection{Solar Cell Characterization}\label{subsubsec:solar_cell_characterization}

\todo[inline]{Process of characterizing solar cell (assembly, test, disassembly)}


\subsubsection{Extraction of Cell Parameters}\label{subsubsec:extraction_of_cell_parameters}

\todo[inline,caption={}]{
    \begin{itemize}
        \item Present results of each cell line
        \item Compare against smith et al for Gen III how accurate and precise
        the cell distribution is
        \item Review parameters that need to be measured empirically (\ac{RS}, \ac{RSH}, etc)
        \item Discussion on how to measure series and shunt resistance
        empirically (refer to Appendix E)
        \item Discussion on how to measure temperature coefficients empirically
        \item Discussion on curve fitting techniques
    \end{itemize}
}


\subsubsection{Modeling Solar Cell Datasets}\label{subsubsec:modeling_solar_cell_datasets}

\todo[inline,caption={}]{
    \begin{itemize}
        \item Discuss python model for modeling cells, iterative solving (refer
        to Appendix F)
        \item Present initial figures showing expected model output
    \end{itemize}
}


\subsubsection{Evaluating Solar Cell Models}\label{subsubsec:evaluating_solar_cell_models}


\section{Modeling Solar Modules}\label{sec:modeling_solar_modules}

After modeling the solar cell, the next layer up in the abstraction chain is
modeling a solar module. A solar module, in a typical configuration, may consist
of several solar cells in series, also called a `string'. For the \ac{LHRs}
vehicle, we place many modules in series with each other before terminating at
our variable load, a \acf{MPPT}. In our configuration, for each module, we place
a `bypass diode' in antiparallel to each module. This is used to provide an
alternative path of current to flow in the event that a solar module is
insufficient for driving the current. This could happen if a single (or several)
solar cell(s) in the module is (are) broken, or shaded, or some combination of
the two.

In this section, we'll focus on the conditions that cause issues with solar
modules, in particular, cell mismatch. We'll look at how cells with differing
operating conditions in series can drag down the efficiency of the module, and
we'll incorporate the bypass diode into the model and investigate the
conditions in which it activates and to what degree it activates.

\subsection{Modeling Photovoltaic Strings}\label{subsec:solar_cell_strings}


\begin{figure}[!htbp]
    \includegraphics[width=\textwidth]{solar_module_model.png}
    \caption{Solar Module Model}
    \label{fig:solar_module_model}
\end{figure}

\subsection{Modeling Bypass Diodes}\label{subsec:bypass_diodes}

\subsection{Evaluation of Solar Module Models}\label{subsec:eval_solar_module_models}

\todo[inline]{Introduction for evaluation of solar module models.}


\subsubsection{Solar Module Test Setup}\label{subsubsec:solar_module_test_setup}

\todo[inline]{
    Refer to cell test setup and modifications for solar modules.
    Discuss applying a power supply on the input of the module.
}


\subsubsection{Solar Module Dataset}\label{subsubsec:solar_module_dataset}

\todo[inline,caption={}]{
    \begin{enumerate}
        \item Discussion of modules assembled, individual cell \ac{I-V} curves
        \item Discussion of lamination process (add reference to appendix for
        full procedure) and expected effect on efficiency
    \end{enumerate}
}


\subsubsection{Modeling Solar Module Datasets}\label{subsubsec:modeling_solar_module_datasets}

\todo[inline]{
    Compare Python model using extracted parameters against module \ac{I-V}
    curves.
}


\subsubsection{Results}\label{subsubsec:solar_module_model_results}

\todo[inline]{
    Holistic evaluation w/ table on statistics of model: e.g. for the model and
    extracted module parameters, what is the overall module accuracy and
    precision?
}


\section{Modeling Solar Arrays}\label{sec:modeling_solar_arrays}

Finally, we take the solar module model developed in the previous section and
combine it with many other modules in series to form a solar array model. In
this section we will focus on the global \ac{I-V} curve generated from the solar
array, and how the overall properties of the curve change under differing
environmental conditions, namely irradiance changes and partial shading. We'll
also look at the temporal properties of a solar array model, such as how long it
takes to effectively change voltage across the solar array.

\subsection{Modeling Shading Effects}\label{subsec:shading_effects}

\newpage
\subsection{Modeling Dynamic Behaviors}\label{subsec:temporal_modeling}

\newpage
\subsection{Evaluation of Solar Array Models}\label{subsec:eval_solar_array_models}

\todo[inline]{Introduction for evaluation of solar array models.}


\subsubsection{Solar Array Test Setup}\label{subsubsec:solar_array_test_setup}

\todo[inline]{Discuss outside array testing.}


\subsubsection{Modeling Solar Array Datasets}\label{subsubsec:modeling_solar_array_datasets}

\todo[inline]{
    Compare Python model using extracted parameters against array \ac{I-V}
    curves.
}


\subsubsection{Results}\label{subsubsec:solar_array_model_results}

\todo[inline]{
    Holistic evaluation w/ table on statistics of model: e.g. for the model and
    extracted array parameters, what is the overall array accuracy and
    precision?
}



\section{Conclusion}\label{sec:modeling_pvs_conclusion}

%TODO: conclusion for Chapter 1
\todo[inline]{Insert conclusion on chapter topics and results.}
