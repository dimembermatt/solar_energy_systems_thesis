\chapter*{Abstract}

Hello world

Existing industrial, commercial, and residential photovoltaic systems leave a
large portion of the potential harvestable power on the table. These losses
exist at various components of the system, starting from the solar cells and
ending with the load or storage; they all add up to reduce the total system
efficiency. This becomes important especially for systems at scale: both
distributed power generation networks (e.g. residential solar) and centralized
power generation networks (e.g. solar farms) can and will operate far and away
from their maximum capacity factor, requiring developers to overbuild or
subsidize their power needs in order to meet target power requirements. This
increases the financial cost, resource consumption, and environmental footprint
of the utility, which factors into the final approval of the project. Therefore,
it is in the best interest of developers, planners, and consumers to understand
how their photovoltaic systems operate, and how they can better utilize or
maximize the available power to best suit their needs and wants.

This work attempts to provide this understanding through holistic modeling and
empirical evalution of photovoltaic systems in a high stakes


It first suggests improvements to
existing, well known photovoltaic models to improve their accuracy against real
photovoltaics in actual operating conditions.


It attempts to evaluate and suggest
improvements to existing models, characterize

identify bottlenecks in energy production processes and components,

and create targeted insights and design methodology to improve output.

The modeling and evaluation are themes and tools that are used hand in
hand in this work;


This modeling
effort is primarily present in the first and third major sections of this work,
starting from the modeling of the photovoltaic-specific hardware and ending at
the modeling of the photovoltaic infrastructure necessary to convert the energy
into useful forms to the end user. In the second and third major sections of the
work, we build off the insights of the priors to propose algorithms and
processes to improve component efficiency.

\todo[inline]{Results and conclusion}


