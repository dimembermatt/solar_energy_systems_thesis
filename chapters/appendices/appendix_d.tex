\chapter{Calibrating the TSL2591 for an AM1.5 Spectra}\label{appendix:tsl2591_calibration}

In order to calibrate the readings from the sensor as a proxy for
the real irradiance experienced by the solar cell, we need to compare it to
either a known source or a reference pyrometer. Another way to interpret the
readings in counts is to convert it into lux; Michael et
al.~\cite{michael_et_al} proposes several methodologies for determining and
verifying a $\si{\lux}$ to $\si{\watt}/\si{\meter}^2$ conversion factor. They
also propose an `engineering rule of thumb', that $120 \si{\lux} =
\si{\watt}/\si{\meter}^2$.

\todo[inline]{Add potential note later on about needing teflon to filter in
saturation conditions - can this fixed by adjusting gain/integration time?}
\todo[inline]{Reference Gacusan's, Burgt's thesis regarding designing low-cost
pyranometer using TSL2591 and TSL2591-like sensors}
