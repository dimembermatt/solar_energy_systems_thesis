\chapter{Introduction}

In order to reach net zero emissions targets set by the \ac{UN} at the
2015 Paris Agreement~\cite{UN_Paris_agreement} before 2050, the \ac{IEA}
estimates that nearly 630 \ac{GW}~\cite{IEA_roadmap} of \ac{PV} energy
generation capacity need to be added annually by 2030. As of 2022, we observed
that at least 175 GW were installed in 2021~\cite{IEA_trends}~\cite{IEA_snapshot}, a 22\% year over year growth. With large policy and
geopolitical tailwinds behind major economies like the United States and Europe,
solar is expected to be one of the, if not the major driver of new energy
generation within the next two decades.

However, in order to achieve this target generation capacity in a sustainable
way, engineers and \ac{PV} designers need to maximize the electrical
efficiency of the overall power system, as opposed to just improving the solar
cell efficiency. According to the \ac{EIA}~\cite{EIA_capacity}, the capacity
factor of \ac{PV}s as an energy source in the United States reached a
monthly maximum of 33.4\% in June of 2022; \textit{capacity factor} is defined
by the \ac{EIA} as a measure of the generated output by the electric generator
versus the maximum possible output. It is clear that system inefficiencies in
\ac{PV} generation provide large constraints, and optimistically, equally
large opportunities, in allowing us to increase our pace towards reaching net
zero carbon emissions by 2050.

This thesis takes a holistic evaluation of the \ac{PV} power generation
system in a unique use case that necessitates maximizing the capacity factor:
solar powered vehicles. We evaluate the modeling, creation, and optimization of
a solar powered vehicle for the University of Texas at Austin's \ac{LHRs} team,
and attempt to identify and address inefficiencies and bottlenecks whose
improvements will help the larger \ac{PV} industry as a whole.

In particular, this thesis will focus on three important and active areas of
development within the \ac{PV} field: solar array modeling and prediction,
\todo{The second area of development may be more generalized then this.}
\hl{solar cell binning processes and heuristics},
and \ac{MPPT} algorithms.
In each of these areas, we look at the state of the art techniques, propose novel
ideas to improve our understanding of the system and its inefficiencies, and see
if we can translate it lateral applications like rooftop solar or industrial
\ac{PV}.\@ Note that in this thesis we refer to photovoltaics and solar without
distinction.

In the first major chapter, \textbf{Modeling Photovoltaics}, this thesis
discusses how can solar cells can be modeled at various abstraction layers, from
idealized cells at standard conditions using the 3-parameter model to
non-idealized cells that incorporate parasitic resistances using the 7-parameter
model. These solar cell models are then evaluated against a dataset of several
hundred solar cell \ac{I-V} and \ac{P-V} curves generated from our custom
testing setup to see how well the model fits real cells at different conditions.
We build upon these models to form larger units of \ac{PV}s, such as solar
modules and solar arrays, which may consist of strings of cells in series with
bypass diodes across them, among other configurations. Some important topics
that are explored using these multi-cell models include \ac{PV} mismatch and
bypass activation. Insights from these topics lead to heuristics that are
proposed in the next chapter, \textbf{Optimizing Photovoltaics}.

The second major chapter, \textbf{Optimizing Photovoltaics}, takes the
aforementioned models and dataset created to propose a process to bin, match,
and combine solar cells and modules, with the end goal of maximizing the
performance of the solar array that will be attached to the solar vehicle. In
this chapter, we propose design criteria, heuristics, and methodologies to
generate designs for the solar vehicle that fit the unique constraints of the
application, which center around the dynamism of the system as it moves in
transit across the real world.

In the third and final major chapter, \textbf{Optimizing Photovoltaic
Systems}, this thesis investigates the operation of the \ac{PV}
system in the context of the solar vehicle. We observe the energy conversion
process from incident light on the solar array to electricity captured by the
\ac{BPS} and present a \ac{PV} system simulator and a suite of \ac{MPPT}
algorithms to minimize energy losses from the aforementioned conversion process.
We demonstrate custom hardware developed by the \ac{LHRs} team and evaluate in
real world settings a select set of \ac{MPPT} algorithms. We compare these
results with existing research and our digital twin model of the solar vehicle,
and finally discuss conclusions from the three chapters that can be translatable
to the wider \ac{PV} industry.

Along with these three major chapters, we also provide a large set of appendices
corresponding to the development of the main body of work in this thesis. Among
them include manufacturing procedures for testing, assembling, and laminating
solar cells into solar modules, schematics and accompanying documentation for
hardware that was used for characterizing and validating parts of the thesis,
software diagrams with relevant open source software repositories developed by
our team, and extra insights into the design of the \ac{LHRs}' photovoltaic
array that are not directly applicable to the major chapters, such as thermal
models performed of the vehicle topshell that influenced our simulation models,
among others.
